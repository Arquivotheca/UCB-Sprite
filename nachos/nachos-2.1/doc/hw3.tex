\documentstyle{article}

\begin{document}

\begin{center}
{\large {\bf NACHOS}}

% \vspace{.2in}
% Tom Anderson\\
% Wayne Christopher\\
% Steven Procter\\

% \vspace{.2in}
% Computer Science 162\\
% March 12, 1992

\vspace{.5in}
Assignment \#3: Multiprogramming\\
Due date: Thursday, April 2, 5:00 p.m.
\end{center}

\vspace{.2in}

The third phase of Nachos is to support multiprogramming.
As in the first two assignments, we give you some of the code
you need; your job is to complete the system and enhance it.

The first step is to read and understand the part of the system
we have written for you.  Our code can run a single user-level `C'
program at a time.  As a test case, we've provided you with 
a trivial user program, `halt'; all halt does is to turn around
and ask the operating system to shut the 
machine down.  First, load the program onto the Nachos disk -- 
`nachos -f -cp test/halt halt'; then run the program -- 
`nachos -x halt'.  As before, trace what happens as the user program
gets loaded, runs, and invokes a system call.

The files for this assignment are:

\begin{description}

\item test3.cc --- test routines for running user programs.

\item addrspace.h, addrspace.cc -- create an address space in which
to run a user program, and load the program from disk.

\item syscall.h -- the system call interface: kernel procedures that 
user programs can invoke.

\item exception.cc -- the handler for system calls and other user-level
exceptions, such as page faults.  In the code we supply, only the 
`halt' system call is supported.

\item pte.h, pte.cc -- page table management routines.  In the code
we supply, we assume that every virtual address is the same as its 
physical address -- this restricts us to running one user program at 
a time.  You will generalize this to allow multiple user programs to
be run concurrently.  We will not ask you to implement virtual 
memory support until in assignment 4; for now, every page must 
be in physical memory.  

\item machine.h, machine.cc -- emulates the part of the machine that
executes user programs: main memory, processor registers, etc.

\item mipssim.cc -- emulates the integer instruction set of a 
MIPS R2/3000 processor.

\item console.h, console.cc -- emulates a terminal device.  A terminal
is like a disk except (i) it is byte, rather than sector, oriented, 
(ii) incoming bytes can be read and written at the same time, and (iii)
bytes arrive asynchronously (as a result of user keystrokes), without 
being explicitly requested.  As with the disk, we emulate the console
input and output as UNIX files (e.g., stdin and stdout).

\end{description}

Until now, all the code you have run on Nachos has been part of the
operating system.  The synchronization code was used only by the kernel,
and the file system operations were invoked from the command line of 
Nachos, rather than by user level programs.  In a
real operating system, the kernel not only uses these services internally,
but allows user-level programs to access them via ``system calls''.

In assignment 2, we gave you a simulated disk that modelled the
behavior of a real disk.  Likewise, in this assignment we are giving
you a simulated CPU that models a real CPU.  In fact, the simulated
CPU is the same as the real CPU (a MIPS chip), but we cannot just run
user programs as regular UNIX processes, because we want complete
control over how many instructions are executed at a time, how the
address spaces work, and how interrupts and exceptions (including
system calls) are handled.

Our simulator can run normal programs compiled from C -- see 
the Makefile in the `test' subdirectory for an example.  The compiled
programs must be converted into Nachos format, using the
program ``coff2flat'' (which we supply), before being loaded onto 
the Nachos disk.  Two caveats: (i) floating point operations are not supported
and (ii) the `main' procedure must be listed first in its C file
(if there are multiple `.o' files, the one containing `main' must be 
listed first on the `ld' command line).  This ensures that `main' 
will be loaded at virtual address 0.  In UNIX, `main' can be located 
anywhere in a program, because a special procedure `START' is located
at a well-known virtual address (and START calls main).

There are four parts to this assignment; each is weighted equally.
If your file system from assignment 2 doesn't work, you may still 
do assignment 3, by replacing calls to the Nachos file system with
direct calls to the UNIX file system; however, 20\% of the grade for
this assignment is based on whether you use your file system code from
assignment 2.

\begin{description}
\item{1.}
Implement system call and exception handling.  You must
support all of the system calls defined in syscall.h, including
the ability to pass arguments to a newly created address space.
We have provided you an assembly-language routine, ``syscall'', to
provide a way of invoking a system call from a C routine (UNIX has
something similar -- try `man syscall').   You'll need to do part 2 of 
this assignment in order to test out the `exec' and `wait' system calls;
the routine `StartProcess' in test3.cc may be of use to you in
implementing the `exec' system call.

Note that you will need to ``bullet-proof'' the Nachos kernel from
user program errors -- there should be nothing a user program can
do to crash the operating system (with the exception of explicitly asking 
the system to halt).  Also, to support the system calls that access 
the console device, you will probably find it helpful to implement 
a ``SynchConsole'' class, that has the same role in managing the physical 
console device as SynchDisk has in managing the physical disk.
``test3.cc'' has the beginnings of a SynchConsole implementation.

\item{2.}
Implement multiprogramming with time-slicing.  The code we have given
you is restricted to running one user program at a time.
You will need to: (a) come up with a way of allocating physical memory 
frames so that multiple programs can be loaded into memory at once, 
(b) provide a way of copying data to/from the kernel from/to the user's 
virtual address space (now that the addresses the user program sees
are not the same as the ones the kernel sees), and (c) use timer interrupts 
to force threads to yield after a certain number of ticks.

Note that scheduler.cc now saves and restores user machine state
on context switches.

\item{3.}
Write a shell and some utility programs.  A shell reads a command 
from the user via the console, then runs the command by invoking
the kernel system call `exec'.
The UNIX program `csh' is an example of a shell.  
Test out your shell and system call handling by writing a couple 
utility programs, such as UNIX `cat' and/or `cp'.

\item{4.}
Optimize Nacho's multiprogramming performance in running the `perftest'
test case (invoke `nachos -x perftest' after loading it into the Nachos 
file system).  This test case starts up a CPU-bound job and an I/O-bound job.
You will probably need to carefully select the time-slice interval;
you may also need to implement some kind of multi-level thread 
scheduling in order to get good performance.

Include in your writeup a journal of your experiences as in assignment 2.
How do certain design choices affect performance?

\end{description}
\end{document}
